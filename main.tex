%-------------------------
% Resume in Latex
% Author : Spencer Wallace
% License : MIT
%------------------------

\documentclass[letterpaper,11pt]{article}

\usepackage{latexsym}
\usepackage[empty]{fullpage}
\usepackage{titlesec}
\usepackage{marvosym}
\usepackage[usenames,dvipsnames]{color}
\usepackage{verbatim}
\usepackage{enumitem}
\usepackage[colorlinks=true,linkcolor=blue,urlcolor=blue]{hyperref}
\usepackage{fancyhdr}
\usepackage[english]{babel}
\usepackage{tabularx}
\usepackage{hyphenat}
\usepackage{fontawesome}
\input{glyphtounicode}


%---------- FONT OPTIONS ----------
% sans-serif
% \usepackage[sfdefault]{FiraSans}
% \usepackage[sfdefault]{roboto}
% \usepackage[sfdefault]{noto-sans}
% \usepackage[default]{sourcesanspro}

% serif
% \usepackage{CormorantGaramond}
% \usepackage{charter}


\pagestyle{fancy}
\fancyhf{} % clear all header and footer fields
\fancyfoot{}
\renewcommand{\headrulewidth}{0pt}
\renewcommand{\footrulewidth}{0pt}

\newcommand{\PaperEntry}[6]{\noindent #1, ``{#2}", \textit{#3} \textbf{#4}, #5 (#6).}
\newcommand{\ArxivEntry}[4]{\noindent #1, ``{#2}", \textit{#3} (#4).}

\newcommand{\ProposalEntry}[3]{#1 \hfill {\em #2} \\ \textit{#3}}
\newcommand{\ConferenceEntry}[2]{#1 \hfill {\em #2}}

% Adjust margins
\addtolength{\oddsidemargin}{-0.5in}
\addtolength{\evensidemargin}{-0.5in}
\addtolength{\textwidth}{1in}
\addtolength{\topmargin}{-.5in}
\addtolength{\textheight}{1.0in}

\urlstyle{same}

\raggedbottom
\raggedright
\setlength{\tabcolsep}{0in}

% Sections formatting
\titleformat{\section}{
  \vspace{-4pt}\scshape\raggedright\large
}{}{0em}{}[\color{black}\titlerule \vspace{-5pt}]

% Ensure that generate pdf is machine readable/ATS parsable
\pdfgentounicode=1

%-------------------------
% Custom commands

\newcommand{\resumeItem}[1]{
  \item\small{
    {#1 \vspace{-2pt}}
  }
}


\newcommand{\resumeSubheading}[4]{
  \vspace{-2pt}\item
    \begin{tabular*}{0.97\textwidth}[t]{l@{\extracolsep{\fill}}r}
      \textbf{#1} & #2 \\
      \textit{\small#3} & \textit{\small #4} \\
    \end{tabular*}\vspace{-7pt}
}


\newcommand{\resumeSubSubheading}[2]{
    \vspace{-2pt}\item
    \begin{tabular*}{0.97\textwidth}{l@{\extracolsep{\fill}}r}
      \textit{\small#1} & \textit{\small #2} \\
    \end{tabular*}\vspace{-7pt}
}


\newcommand{\resumeEducationHeading}[6]{
  \vspace{-2pt}\item
    \begin{tabular*}{0.97\textwidth}[t]{l@{\extracolsep{\fill}}r}
      \textbf{#1} & #2 \\
      \small#3 & \textit{\small #4} \\
      \textit{\small#5} & \textit{\small #6} \\
    \end{tabular*}\vspace{-5pt}
}


\newcommand{\resumeProjectHeading}[2]{
    \vspace{-2pt}\item
    \begin{tabular*}{0.97\textwidth}{l@{\extracolsep{\fill}}r}
      \small#1 & #2 \\
    \end{tabular*}\vspace{-7pt}
}


\newcommand{\resumeOrganizationHeading}[4]{
  \vspace{-2pt}\item
    \begin{tabular*}{0.97\textwidth}[t]{l@{\extracolsep{\fill}}r}
      \textbf{#1} & \textit{\small #2} \\
      \textit{\small#3}
    \end{tabular*}\vspace{-7pt}
}

\newcommand{\resumeSubItem}[1]{\resumeItem{#1}\vspace{-4pt}}

\renewcommand\labelitemii{$\vcenter{\hbox{\tiny$\bullet$}}$}

\newcommand{\resumeSubHeadingListStart}{\begin{itemize}[leftmargin=0.15in, label={}]}
\newcommand{\resumeSubHeadingListEnd}{\end{itemize}}
\newcommand{\resumeItemListStart}{\begin{itemize}}
\newcommand{\resumeItemListEnd}{\end{itemize}\vspace{-5pt}}

%-------------------------------------------
%%%%%%  RESUME STARTS HERE  %%%%%%%%%%%%%%%%%%%%%%%%%%%%


\begin{document}

%---------- HEADING ----------

\begin{center}
    \textbf{\Huge \scshape Spencer Wallace} \\ \vspace{3pt}
    \small
    \faMobile \hspace{.5pt} \href{tel:15204614480}{520 461 4480}
    $|$
    \faAt \hspace{.5pt} \href{mailto:spencerw530@gmail.com}{spencerw530@gmail.com}
    $|$
    \faGlobe \hspace{.5pt} \href{https://spencerw.github.io}{Website}
    $|$
    \faGithub \hspace{.5pt} \href{https://github.com/spencerw}{GitHub}
    $|$
    \faLinkedinSquare \hspace{.5pt} \href{https://www.linkedin.com/in/scwallace7}{LinkedIn}
    \faMapMarker \hspace{.5pt} \href{https://maps.app.goo.gl/bhFhbfkGV9YasvhG8}{San Diego, CA}
\end{center}



%----------- EDUCATION -----------

\section{Education}
  \vspace{3pt}
  \resumeSubHeadingListStart
    
    \resumeEducationHeading
      {University of Washington
      }{Seattle, Washington}
      {PhD Astronomy}{Oct 2015 \textbf{--} August 2023}{Thesis: Planetesimal Accretion in the Solar System and Beyond}{}
    
    \resumeEducationHeading
      {University of Arizona}
      {Tucson, Arizona}
      {BS Computer Science, Astronomy and Physics}{Aug 2009 \textbf{--} May 2014}{Thesis: Turbulent Entrainment in 1D Stellar Evolution Models}{}
    
  \resumeSubHeadingListEnd



%----------- RESEARCH EXPERIENCE -----------

\section{Work Experience}
  \vspace{3pt}
  \resumeSubHeadingListStart
  
    \resumeSubheading
      {Postdoctoral Researcher}{}
      {University of Washington, Astronomy Department (Remote, Full-Time)}{September 2023 \textbf{--} Present}
        \resumeItemListStart
        	   \resumeItem{Led the deployment of the tree-based N-body code {\sc \href{https://github.com/N-BodyShop/changa}{ChaNGa}} on the new Grace Hopper CPU-GPU nodes at the Texas Advanced Computing Center}
	   \resumeItem{Performed memory and performance profiling, along with debugging of ChaNGa on the new hardware using a suite of C++ and CUDA-based tools}
	   \resumeItem{Collaborated with a wide range of specialists facing similar algorithm challenges to improve the performance of our code, including computer scientists, physicists and molecular biologists}
        \resumeItemListEnd
        
     \resumeSubheading
      {Research Assistant}{}
      {University of Washington, Astronomy Department (Full-Time)}{October 2019 \textbf{--} August 2023}
        \resumeItemListStart
        	   \resumeItem{Designed, proposed and executed a \$450,000 National Science Foundation-funded research project to simulate the formation of Earth-like planets using a high-performance computing cluster, which resulted in three first-author publications}
	   \resumeItem{Implemented a tree-based collision detection module in the N-body code  {\sc \href{https://github.com/N-BodyShop/changa}{ChaNGa}} to track the growth of solid bodies, allowing me to run the first-ever simulation of planet formation using realistic-sized objects}
	   \resumeItem{Built a pipeline to artificially generate simulation results using a score-based diffusion model, saving over 900,000 CPU hours of computation}
	   \resumeItem{Trained and mentored 5 undergraduate students while they ran and analyzed simulations on a local HPC cluster}
	   \resumeItem{Produced a coauthored \href{https://doi.org/10.1109/IPDPS53621.2022.00079}{IEEE publication} while helping to develop {\sc \href{https://paratreet.github.io/}{paratreet}}, a toolkit for quickly testing and tuning parallel tree traversal algorithms}
        \resumeItemListEnd
        
       \resumeSubheading
      {Teaching Assistant}{}
      {University of Washington, Astronomy Department (20 hrs/wk)}{October 2015 \textbf{--} September 2019}
        \resumeItemListStart
        	   \resumeItem{Led weekly discussion sections, graded exams, and occasionally designed homework assignments and performed guest lectures for undergraduate astronomy classes, some of which involved teaching basic data analysis and numerical modeling skills using Python and Excel}
	   \resumeItem{Collaborated with a team of other teaching assistants to ensure exams and assignments were graded consistently and fairly}
        \resumeItemListEnd
        
       \resumeSubheading
      {Public Program Specialist}{}
      {National Optical Astronomy Observatory (20 hrs/wk)}{September 2014 \textbf{--} April 2015}
        \resumeItemListStart
        	   \resumeItem{Led night time educational astronomy programs for the general public at Kitt Peak National Observatory}
	   \resumeItem{Delivered lectures and answered questions for groups of guests during tours of the telescope facilities, which included observing the night sky with both binoculars and larger optical telescopes}
        \resumeItemListEnd        
  \resumeSubHeadingListEnd
  
  %----------- SKILLS -----------

\section{Skills}
  \vspace{2pt}
  \resumeSubHeadingListStart
    \small{\item{
        \textbf{Programming:}{ C++, Python, NumPy, Pandas, PyTorch, scikit-learn, matplotlib, Seaborn, SQL \\ \vspace{3pt}
        
        \textbf{Communication:}{ 3 first-authored publications, 2 co-authored publications, 7 conference talks, 3 conference posters, 20 pop-sci articles published through \href{https://astrobites.org/}{astrobites}} \\ \vspace{3pt}
        
        \textbf{Leadership:}{ Worked on 5 separate science collaboration teams, Mentored and directed research for 6 undergraduate students} \\ \vspace{3pt}
    }}}
  \resumeSubHeadingListEnd
  
  %----------- Publications -----------
  
  \section{Publications}
  
  \vspace{2pt}
  \resumeSubHeadingListStart
    \small{\item{

\PaperEntry{\underline{Wallace, S.}, Quinn, T.}{\href{https://doi.org/10.3847/1538-4357/ace89c}{Planetesimal accretion at short orbital periods}}{}{ApJ} {954(1):61}{2023} \\ \vspace{6pt}

\PaperEntry{Hutter, J., Szaday, J., Choi, J., Liu, S., Kale, L., \underline{Wallace, S.}, Quinn, T.}{\href{https://doi.org/10.1109/IPDPS53621.2022.00079}{ParaTreeT: A Fast, General Framework for Spatial Tree Traversal}}{}{IEEE IPDPS} {762-772}{2022} \\ \vspace{6pt}

\PaperEntry{\underline{Wallace, S.}, Quinn, T., Boley, A.}{\href{https://doi.org/10.1093/mnras/stab792}{Collision rates of planetesimals near mean-motion resonances}}{}{MNRAS} {503(4):5409–5424}{2021} \\ \vspace{6pt}

\PaperEntry{\underline{Wallace, S.} and Quinn, T.}{\href{https://doi.org/10.1093/mnras/stz2284}{N-body simulations of terrestrial planet growth with resonant dynamical friction}}{}{MNRAS}{489(2):2159–2176}{2019}  \\ \vspace{6pt}

\PaperEntry{Kim, J., Agertz, O., Teyssier, R., Butler, M., Ceverino, D., Choi, J., Feldmann, R., Keller, B., Lupi, A., Quinn, T., Revaz, Y., \underline{Wallace, S} (and 31 more)}{\href{https://doi.org/10.3847/1538-4357/833/2/202}{The AGORA High-resolution Galaxy Simulations Comparison Project. II. Isolated Disk Test}}{}{ApJ}{833(2):202}{2016} \\ \vspace{6pt}
        
    }}
  \resumeSubHeadingListEnd
  
    %----------- Proposals -----------
    
\section{Successful Proposals}
  \resumeSubHeadingListStart
    \small{\item{
 
\ProposalEntry{Contributed to Successful NSF Grant}{2020}{In Situ Formation of Short Period Terrestrial Planets} \\ \vspace{6pt}

\ProposalEntry{Contributed to Successful XSEDE Computing Proposal}{2019, 2020, 2021}{N-body Simulations: Planets to Cosmology} \vspace{6pt}

 }}     
  \resumeSubHeadingListEnd
  
\section{Conference Presentations}
  \resumeSubHeadingListStart
    \small{\item{
    
 \ConferenceEntry{24th Charm++ Workshop, Contributed Talk}{April 2024}\\
 \ConferenceEntry{241st AAS Meeting, Dissertation Talk}{January 2023} \\ 
 \ConferenceEntry{Exoplanets in Our Backyard II, Poster}{November 2022} \\ 
 \ConferenceEntry{2nd N-body Shop Collaboration Meeting, Contributed Talk}{June 2022} \\
 \ConferenceEntry{53rd AAS Division for Dynamical Astronomy Meeting, Contributed Talk}{May 2022} \\ 
 \ConferenceEntry{TESS Science Conference II, Poster}{August 2021} \\ 
 \ConferenceEntry{Sagan Exoplanet Summer Workshop, Poster}{July 2021} \\ 
 \ConferenceEntry{52nd AAS Division for Dynamical Astronomy Meeting, Contributed Talk}{May 2021} \\ 
 \ConferenceEntry{1st N-body Shop Collaboration Meeting, Contributed Talk}{January 2021} \\ 
 \ConferenceEntry{50th AAS Division for Dynamical Astronomy Meeting, Contributed Talk}{June 2019} \\ 
  \ConferenceEntry{49th AAS Division for Dynamical Astronomy Meeting, Contributed Talk}{April 2018} \\ 
  \ConferenceEntry{223rd AAS Meeting, Poster}{January 2014} \\ 
 
 \vspace{6pt}

 }}     
  \resumeSubHeadingListEnd
  
\end{document}
