%-------------------------
% Resume in Latex
% Author : Spencer Wallace
% License : MIT
%------------------------

\documentclass[letterpaper,11pt]{article}

\usepackage{latexsym}
\usepackage[empty]{fullpage}
\usepackage{titlesec}
\usepackage{marvosym}
\usepackage[usenames,dvipsnames]{color}
\usepackage{verbatim}
\usepackage{enumitem}
\usepackage[colorlinks=true,linkcolor=blue,urlcolor=blue]{hyperref}
\usepackage{fancyhdr}
\usepackage[english]{babel}
\usepackage{tabularx}
\usepackage{hyphenat}
\usepackage{fontawesome}
\input{glyphtounicode}


%---------- FONT OPTIONS ----------
% sans-serif
% \usepackage[sfdefault]{FiraSans}
% \usepackage[sfdefault]{roboto}
% \usepackage[sfdefault]{noto-sans}
% \usepackage[default]{sourcesanspro}

% serif
% \usepackage{CormorantGaramond}
% \usepackage{charter}


\pagestyle{fancy}
\fancyhf{} % clear all header and footer fields
\fancyfoot{}
\renewcommand{\headrulewidth}{0pt}
\renewcommand{\footrulewidth}{0pt}

% Adjust margins
\addtolength{\oddsidemargin}{-0.5in}
\addtolength{\evensidemargin}{-0.5in}
\addtolength{\textwidth}{1in}
\addtolength{\topmargin}{-.5in}
\addtolength{\textheight}{1.0in}

\urlstyle{same}

\raggedbottom
\raggedright
\setlength{\tabcolsep}{0in}

% Sections formatting
\titleformat{\section}{
  \vspace{-4pt}\scshape\raggedright\large
}{}{0em}{}[\color{black}\titlerule \vspace{-5pt}]

% Ensure that generate pdf is machine readable/ATS parsable
\pdfgentounicode=1

%-------------------------
% Custom commands

\newcommand{\resumeItem}[1]{
  \item\small{
    {#1 \vspace{-2pt}}
  }
}


\newcommand{\resumeSubheading}[4]{
  \vspace{-2pt}\item
    \begin{tabular*}{0.97\textwidth}[t]{l@{\extracolsep{\fill}}r}
      \textbf{#1} & #2 \\
      \textit{\small#3} & \textit{\small #4} \\
    \end{tabular*}\vspace{-7pt}
}


\newcommand{\resumeSubSubheading}[2]{
    \vspace{-2pt}\item
    \begin{tabular*}{0.97\textwidth}{l@{\extracolsep{\fill}}r}
      \textit{\small#1} & \textit{\small #2} \\
    \end{tabular*}\vspace{-7pt}
}


\newcommand{\resumeEducationHeading}[6]{
  \vspace{-2pt}\item
    \begin{tabular*}{0.97\textwidth}[t]{l@{\extracolsep{\fill}}r}
      \textbf{#1} & #2 \\
      \textit{\small#3} & \textit{\small #4} \\
      \textit{\small#5} & \textit{\small #6} \\
    \end{tabular*}\vspace{-5pt}
}


\newcommand{\resumeProjectHeading}[2]{
    \vspace{-2pt}\item
    \begin{tabular*}{0.97\textwidth}{l@{\extracolsep{\fill}}r}
      \small#1 & #2 \\
    \end{tabular*}\vspace{-7pt}
}


\newcommand{\resumeOrganizationHeading}[4]{
  \vspace{-2pt}\item
    \begin{tabular*}{0.97\textwidth}[t]{l@{\extracolsep{\fill}}r}
      \textbf{#1} & \textit{\small #2} \\
      \textit{\small#3}
    \end{tabular*}\vspace{-7pt}
}

\newcommand{\resumeSubItem}[1]{\resumeItem{#1}\vspace{-4pt}}

\renewcommand\labelitemii{$\vcenter{\hbox{\tiny$\bullet$}}$}

\newcommand{\resumeSubHeadingListStart}{\begin{itemize}[leftmargin=0.15in, label={}]}
\newcommand{\resumeSubHeadingListEnd}{\end{itemize}}
\newcommand{\resumeItemListStart}{\begin{itemize}}
\newcommand{\resumeItemListEnd}{\end{itemize}\vspace{-5pt}}

%-------------------------------------------
%%%%%%  RESUME STARTS HERE  %%%%%%%%%%%%%%%%%%%%%%%%%%%%


\begin{document}

%---------- HEADING ----------

\begin{center}
    \textbf{\Huge \scshape Spencer Wallace} \\ \vspace{3pt}
    \small
    \faMobile \hspace{.5pt} \href{tel:15204614480}{520 461 4480}
    $|$
    \faAt \hspace{.5pt} \href{mailto:spencerw530@gmail.com}{spencerw530@gmail.com}
    $|$
    \faLinkedinSquare \hspace{.5pt} \href{https://www.linkedin.com/in/scwallace7}{LinkedIn}
    $|$
    \faGithub \hspace{.5pt} \href{https://github.com/spencerw}{GitHub}
    $|$
    \faGlobe \hspace{.5pt} \href{https://spencerw.github.io}{Portfolio}
    $|$
    \faMapMarker \hspace{.5pt} \href{https://goo.gl/maps/x8PZ2ZAZNJgSbc5K9}{Tucson, AZ}
\end{center}



%----------- EDUCATION -----------

\section{Education}
  \vspace{3pt}
  \resumeSubHeadingListStart
    
    \resumeEducationHeading
      {University of Washington
      }{Seattle, Washington}
      {PhD Astronomy}{Oct 2015 \textbf{--} Summer 2023 (Expected, flexible)}{}{}
    
    \resumeSubheading
      {University of Arizona}
      {Tucson, Arizona}
      {BS Computer Science, Astronomy and Physics}{Aug 2009 \textbf{--} May 2014}
    
  \resumeSubHeadingListEnd



%----------- RESEARCH EXPERIENCE -----------

\section{Work Experience}
  \vspace{3pt}
  \resumeSubHeadingListStart
  
    \resumeSubheading
      {Simulating the assembly of terrestrial planets}{}
      {University of Washington, Astronomy Department}{Jan 2018 \textbf{--} Present}
        \resumeItemListStart
        	    \resumeItem{Designed, proposed and executed a \$450,000 research grant project to model the planet formation process (Python, Numpy, Pandas, Git)}
            \resumeItem{Tested, debugged and ran highly parallel simulations across multiple compute nodes on a variety of national supercomputers (C++, Git, Bash)}
            \resumeItem{Extended a large-scale hydrodynamics code to model collisions between solid bodies (C++, Git)}
            \resumeItem{Created and developed set of analysis tools ({\sc \href{https://github.com/spencerw/CollisionTools}{KeplerOrbit}}, {\sc \href{https://github.com/spencerw/KeplerOrbit}{CollisionTools}}) to track the evolution of large collections of particles (Python, Numpy, Pandas, Git)}
            \resumeItem{Led weekly meetings to train and mentor undergraduate researchers to use python data analysis tools, develop modules for our N-body code, and run simulations}
        \resumeItemListEnd
    
    \resumeSubheading
      {Detecting stellar flares with the Transiting Exoplanet Survey Satellite}{}
      {University of Washington, DIRAC Institute}{Jun 2019 \textbf{--} Jun 2020}
        \resumeItemListStart
        	    \resumeItem{Built a time-series analysis pipeline to find over 7,000 new flare stars in NASA space telescope data (Python, Numpy, Pandas, Git)}
	    \resumeItem{Obtained new rotation period measurements for over 80,000 stars using a gaussian process model}
        \resumeItemListEnd
        
      \resumeSubheading
      {Data synthesis from N-body simulations}{}
      {University of Washington, eScience Institute}{Jan 2023 \textbf{--} Present}
        \resumeItemListStart
        	   \resumeItem{Developed a pipeline to construct initial conditions for planet formation models by training a generative adversarial network (GAN) on existing results (Python, PyTorch, Pandas, Numpy, Git)}
            \resumeItem{Saved over 900,000 CPU hours of computation by using the GAN to skip the first phase of our simulations}
        \resumeItemListEnd
        
        \resumeSubheading
      {Exploring parallel algorithms for spatial tree traversal}{}
      {University of Illinois Urbana-Champaign, Computer Science Department}{Jun 2019 \textbf{--} Apr 2021}
        \resumeItemListStart
            \resumeItem{Participated in a interdisciplinary collaboration to develop {\sc \href{https://paratreet.github.io/}{paratreet}}, a toolkit for quickly testing and tuning parallel spatial tree traversal algorithms (C++, Python, Bash, Git)}
            \resumeItem{Worked with a team of computer scientists to apply and test their algorithms on a number of real-world astronomy applications and reduce force calculation times for our simulations by a factor of 30}
        \resumeItemListEnd
        
       % \resumeSubheading
      %{Verifying the robustness of galaxy simulation codes}{}
      %{University of Washington, Astronomy Department}{Oct 2015 \textbf{--} Dec 2016}
        %\resumeItemListStart
           % \resumeItem{Ran and analyzed a set of hydrodynamics simulations to assess the scientific validity of the most commonly used galaxy simulation codes (C++, Python, Numpy, Bash, Git)}
            %\resumeItem{Worked with a mult-national team from over thirty institutions to highlight and understand and reconcile differences between state-of-the-art galaxy simulation codes asked to solve an identical problem}
        %\resumeItemListEnd
    
    \resumeSubheading
      {Graduate teaching assistant}{}
      {University of Washington, Astronomy Department}{Oct 2015 \textbf{--} Jun 2020}
        \resumeItemListStart
            \resumeItem{Led weekly discussion sections, graded assignments, preformed lectures and designed homework exercises for undergraduate students}
            \resumeItem{Collaborated with a team of other teaching assistants to ensure assignments, quizzes and exams were graded consistently and fairly}
        \resumeItemListEnd
        
  \resumeSubHeadingListEnd
  
  %----------- SKILLS -----------

\section{Skills}
  \vspace{2pt}
  \resumeSubHeadingListStart
    \small{\item{
        \textbf{Programming:}{ C++, Python, NumPy, Pandas, PyTorch, scikit-learn, matplotlib, Seaborn, SQL \\ \vspace{3pt}
        
        \textbf{Communication:}{ 3 first-authored publications, 3 co-authored publications, 7 conference talks, 3 conference posters, 20 pop-sci articles published through \href{https://astrobites.org/}{astrobites}} \\ \vspace{3pt}
        
        \textbf{Leadership:}{ Worked on 5 separate science collaboration teams, Mentored and directed research for 6 undergraduate students} \\ \vspace{3pt}
    }}}
  \resumeSubHeadingListEnd
\end{document}
